\chapter{Аналитический раздел}

\textbf{Цель} данной работы - протестировать web-приложение платформа для проведения футбольных турниров.

\section{Описание тестируемой системы}

\textbf{Web-приложение} платформа для проведения футбольных турниров состоит из следующих частей:
\begin{enumerate}
	\item сервер
	\item база данных
	\item клиентская часть
\end{enumerate}

\textbf{Сервер} написан на языке JavaScript с использованием программной платформы NodeJS. Основное предназначение - обработка http запросов.

\textbf{База данных} - нереляционная СУБД mongoDB. Обращение к базе данных осуществляется с помощью библиотеки Mongoose. Mongoose - обертка, позволяющая создавать удобные и функциональные схемы документов. 

\textbf{Клиентская часть} написана на языке JavaScript с использованием библиотеки ReactJS.

Модели, описывающие объекты веб-приложения/базы данных:
\begin{enumerate}
	\item federation
	\item match
	\item stage
	\item team
	\item tournament
	\item user
	\item vuser
\end{enumerate}

\textbf{app.js} - главный файл тестируемого веб приложения.

В файле \textbf{app.js} подключаются обработчки http запросов, описанных в файлах папки \textbf{routes}:

\begin{enumerate}
	\item federation.js - обработка запросов, связанных с объектом федерация
	\item match.js - обработка запросов, связанных с объектом матч
	\item stage.js - обработка запросов, связанных с объектом этап
	\item team.js - обработка запросов, связанных с объектом команда
	\item tournament.js - обработка запросов, связанных с объектом турнир
	\item vuser.js - обработка запросов, связанных с объектом виртуальный пользователь
	\item referee.js - обработка запросов, связанных с взаимодействием с мобильным приложением
	\item main.js - обработка остальных запросов 
\end{enumerate}

В мобильном приложении авторизуется пользователь и получает список матчей, на которые он назначен судить. Судейство матча означает оправку на сервер следующих событий:
\begin{enumerate}
	\item MATCH\_STARTED - начало матча
	\item MATCH\_FINISHED - конец матча
	\item TIME\_STARTED - начало периода
	\item TIME\_FINISHED - конец периода
	\item GOAL - гол
	\item OWN\_GOAL - автогол
	\item YELLOW\_CARD - желтая карточка
	\item RED\_CARD - красная карточка
	\item MIN - просто прислана текущая минута матча
	\item ASSIST - прислано событие с игроком - который отдал голевой пас.
\end{enumerate}

Каждое сообщение, посылаемое на сервер, должно содержать следующие поля:
 \begin{enumerate}
	\item idAction - порядковый номер события в матче 
	\item idMatch - id матча 
	\item minute - минута матча, на которой произошло событие
	\item idEvent - тип события, описанные выше
\end{enumerate}

\section{Рассматриваемые виды тестирования}

В данной работе будут рассмотрены следующие виды тестирования:
\begin{enumerate}
	\item интеграционное
	\item регрессионное
	\item функциональное
\end{enumerate}


\textbf{Интеграционное тестирование} предназначено для проверки связи между компонентами, а также взаимодействия с различными частями системы (операционной системой, оборудованием либо связи между различными системами).

\textbf{Регрессионное тестирование} - это вид тестирования направленный на проверку изменений, сделанных в приложении или окружающей среде (починка дефекта, слияние кода, миграция на другую операционную систему, базу данных, веб сервер или сервер приложения), для подтверждения того факта, что существующая ранее функциональность работает как и прежде.

\textbf{Функциональное тестирование} рассматривает заранее указанное поведение и основывается на анализе спецификаций функциональности компонента или системы в целом. 
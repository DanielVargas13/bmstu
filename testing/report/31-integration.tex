\section{Интеграционное тестирование}

В данной части работы будет осуществлено тестирование модулей сервера, обрабатывающие http запросы, взаимодействующие с базой данных, в которых импортированы модели. 

\subsection{Тестирование модуля vuser.js}

\textbf{vuser.js} - набор методов для обработки запросов, связанных с объектом виртуальный пользователь.

В данном модуле описаны методы-обработчики:
\begin{enumerate}
	\item \textbf{POST /get} - получение информации о виртуальном пользователе по заданном id
\end{enumerate}

Код модуля \textbf{vuser.js} небольшой и представлен в листинге \ref{lst:vuserjs}

\begin{lstlisting}[caption={Код метода обработчика урла /get}, label={lst:vuserjs}]
var express = require('express');
var router = express.Router();
var Vuser = require('../models/vuser');

router.post('/get', function(req, res, next) {
    if (!req.body.id) {
        return res.status(400).json(null);
    }

    Vuser.findById(req.body.id, (err, vuser) => {
        if(err || !vuser) {
            return res.status(404).json(null);
        }

        return res.status(200).json(vuser);
    });
});

module.exports = router;
\end{lstlisting}

\begin{table}[h] 
\caption{\label{tab:vuserjs}Тестирование модуля vuser.js}
\begin{center}
\begin{tabular}{|l|p{10cm}|}
\hline
\multicolumn{2}{|c|}{1. запрос юзера по существующиму id} \\
\hline
Запрос & \{id: 12345\} \\
Ожидаемый ответ & 200 OK $\{\cdots\}$ \\
\hline
\multicolumn{2}{|c|}{2. запрос юзера по несуществующиму id} \\
\hline
Запрос & \{id: 67890\} \\
Ожидаемый ответ & 404 NOT FOUND: \{null\} \\
\hline
\multicolumn{2}{|c|}{3. запрос юзера c отсутствующим полем id в запросе} \\
\hline
Запрос & \{\} \\
Ожидаемый ответ & 400 BAD REQUEST: \{null\} \\
\hline
\end{tabular}
\end{center}
\end{table} 

\subsection{Тестирование модуля referee.js}
\label{subsec:referee-test}

\textbf{referee.js} - набор методов для обработки запросов, связанных с взаимодействием с мобильным приложением.

В данном модуле описаны методы-обработчики:
\begin{enumerate}
	\item \textbf{POST /get-my-matches} возвращает список матчей, на которые назначен судить заданный пользователь
	\item \textbf{POST /set-info} получает событие, произошедшее в заданном матче с описанием события.
\end{enumerate}

\begin{table}[h] 
\caption{\label{tab:vuserjs}Тестирование модуля referee.js}
\begin{center}
\begin{tabular}{|l|p{10cm}|}

\hline
\multicolumn{2}{|c|}{1. запрос списка матчей по существующему id пользователя} \\
\hline
Запрос & \{id: 12345\} \\
Ожидаемый ответ & 200 OK $\{\cdots\}$ \\
\hline

\multicolumn{2}{|c|}{2. запрос списка матчей по несуществующиму id} \\
\hline
Запрос & \{id: 67890\} \\
Ожидаемый ответ & 404 NOT FOUND: \{null\} \\
\hline

\multicolumn{2}{|c|}{3. запрос списка матчей c отсутствующим полем id в запросе} \\
\hline
Запрос & \{\} \\
Ожидаемый ответ & 400 BAD REQUEST: \{null\} \\
\hline

\multicolumn{2}{|c|}{4. отправка события MATCH\_STARTED в существующем матче} \\
\hline
Запрос & \{idMatch: match.id,
                idEvent: Match.EVENT.MATCH\_STARTED.name,
                idAction: 0,
                minute: 0\} \\
Ожидаемый ответ & 200 OK: \{null\} \\
Ожидаемые действие & Заданный матч изменяет статус на RUNNING \{null\} \\
\hline

\multicolumn{2}{|c|}{5. отправка события MATCH\_FINISHED в существующем матче} \\
\hline
Запрос & \{idMatch: match.id,
                idEvent: Match.EVENT.MATCH\_FINISHED.name,
                idAction: 0,
                minute: 90\} \\
Ожидаемый ответ & 200 OK: \{null\} \\
Ожидаемые действие & Заданный матч изменяет статус на FINISHED \{null\} \\
\hline


\multicolumn{2}{|c|}{6. отправка любого события по несуществующиму id} \\
\hline
Запрос & \{idMatch: "---",
                idEvent: "---",
                idAction: 0,
                minute: 0\} \\
Ожидаемый ответ & 404 NOT FOUND: \{null\} \\
\hline

\multicolumn{2}{|c|}{7. отправка любого события c отсутствующим полем id в запросе} \\
\hline
Запрос & \{\} \\
Ожидаемый ответ & 400 BAD REQUEST: \{null\} \\
\hline

\end{tabular}
\end{center}
\end{table} 